%% 
%% Created in 2018 by Martin Slapak
%%
%% Based on file for NRP report LaTeX class by Vit Zyka (2008)

\documentclass[hidelinks, english]{mvi-report}

\usepackage[utf8]{inputenc}
\usepackage{url}
\usepackage{mathtools}

\usepackage{graphicx}
\usepackage{blindtext}
\usepackage{dirtree}
\usepackage[export]{adjustbox}
\usepackage{array}
\usepackage{bigstrut}
\usepackage{booktabs}
\usepackage{float}
\usepackage{subfigure}
\usepackage[all]{hypcap}
\usepackage[bottom]{footmisc}
\usepackage{caption}
\usepackage{listings}
\usepackage{cleveref}
\usepackage{mathtools}

\graphicspath{{img/}}

\title{Automated machine learning}

\author{Yevhen Kuzmovych}
\affiliation{ČVUT - FIT}
\email{kuzmoyev@fit.cvut.cz}


\newcommand{\subimage}[3][1]{
\subfigure{
    \includegraphics[valign=c, width=#1\textwidth]{#2.#3}
}
}

\newcommand{\smplimage}[3][1]{
\centerline{
    \includegraphics[width=#1\textwidth]{#2.#3}
}
}

\newcommand{\image}[4][1]{
\begin{figure}[H]
    \smplimage[#1]{#2}{#3}
    \caption{#4}
    \label{fig:#2}
\end{figure}
}




\begin{document}

\maketitle

%%%%%%%%%%%%%%%%%%%%%%%%%%%%%%%%%%%%%%%%%%%%%%%%%%%%%%%%%%%%%%%%%%%%%%%%%%%%%%%%
\section{Introduction}

One of the inalienable parts of the data analyst work is selection of the appropriate predictive algorithm for
the given task. In most cases, this part comes down to testing set of selected algorithms on the given dataset or its
subset and selecting the one with the best performance. This process can be automated and improved with prediction of
algorithm quality.

Assuming that each dataset has some hidden properties that could indicate tendency of some algorithms to perform
better than the others, it should be possible to extract those properties and predict algoritms' quality based on them.

There were many attempts that tried to select appropriate meta-features
\cite{sampling-based-relative-landmarks}\cite{statlog}\cite{meta-learning-for-algorithm-selection}. In the framework of
this project, simply obtainable meta-featres used in the StatLog project\cite{statlog} will be combined with landmarks
and relative landmarks described in \textit{Sampling-Based Relative Landmarks: Systematically Test-Driving Algorithms
Before Choosing}\cite{sampling-based-relative-landmarks} to predict algorithms quality.


%%%%%%%%%%%%%%%%%%%%%%%%%%%%%%%%%%%%%%%%%%%%%%%%%%%%%%%%%%%%%%%%%%%%%%%%%%%%%%%%
\section{Methods}

Output library will be able to extract meta-features from the given data and build predictive pipeline using train data.
Algorithm will try to predict performance of each used model, test them in the predicted order, choose the best one and
fit to the train data in a limited time.

Whole process consists of the following parts.

\subsection{Preprocessing}
For this project, only required preprocessing techniques were used:

\begin{itemize}
    \item \textbf{Filling missing data} (because most of the tested models require complete data). NaNs are filled with
    the means in numerical coulmns and most frequent values in categorical.
    \item \textbf{Encoding categorical data}. Nominal data is encoded using \textit{one hot encoding}. Ordinal features
    can be specified with the needed order, labels are then encoded with natural numbers.
    \item \textbf{Dropping constant columns}.
    \item \textbf{Scaling}. Standard scaling of numerical data.
\end{itemize}

Implementation is parameterized and can be easely extended with the other techniques.

\subsection{Meta-data collection}

Meta-features of the given dataset are collected for prediction of the quality of the used models. Collected
meta-features are described in table \ref{fig:meta-features}.

\begin{figure*}[t]
\center
    \begin{tabular}[c]{l l}
    \textbf{Simple}   & \\
    \hline
    NExamples         & Number of examples                      \\
    NFeatures         & Number of features                      \\
    NBinary           & Number of binary features               \\
    NCategorical      & Number of categorical features          \\
    NNumerical        & Number of numerical features            \\
    NExamplesWithNANs & Number of examples with missing values  \\
    NFeaturesWithNANs & Number of features with missing values  \\
    NClasses          & Number of classes (in classification)   \\
    \hline
    \hline
    \textbf{Statistical}   & \\
    \hline
    STDRatio          & Geometric mean of columns standard deviations   \\
    CorrelationMean   & Mean of columns correlation values              \\
    KurtosisMean      & Mean of columns kurtosis values                 \\
    SkewnessMean      & Mean of columns skewness values                 \\
    YImbalance        & STD of number of classes/bins of output column  \\
    YStd              & STD of output column (in regression)            \\
    \hline
    \hline
    \textbf{Relative landmarks}   & \\
    \hline
    rl $i$              & Relative landmark of the model $i$ \\
    rl \dots            & \\
    \end{tabular}

\caption{Collected meta-features}
\label{fig:meta-features}
\end{figure*}

\subsection{Models evaluation and selection}
In model quality evaluation there are two primary characteristics: \textbf{accuracy} and \textbf{processing time}.
Considering this characteristics, models quality is evaluated using so-called \textit{Adjusted Ratio of Ratios}(ARR)
that combines models accuracy and processing time to assess relative performance among other models. \textit{In ARR
the compromise between the two criteria is given by user in the form "the amount of accuracy I'm willing to trade for
a 10 times speed-up is X\%"}\cite{sampling-based-relative-landmarks}. So for two given algorithms $i$ and $j$ on
the data set $d$ the ARR computed as follows:
$$ ARR^d_{ij} = \dfrac{\dfrac{A^d_i}{A^d_j}}{1+\log{(\dfrac{T^d_i}{T^d_j})}*X} $$
where $A^d_i$ is accuracy of the model $i$ on the data set~$d$ and $T^d_i$ is its processing time.

Accuracy in classification problems computed simply as a ratio of the number of correctly classified examples to
the number of total examples: $ A^d_i = \dfrac{C}{N} $. Accuracy for regression problems is computed as:
$$ A^d_i = 1 - \dfrac{RMSE_i}{\max_{j}RMSE_j} $$

Now using computed ARRs, we can generate realtive landmarks for each of $n$ models:
$$ rl^d_i = \dfrac{\sum_{j \neq i}ARR^d_{ij}}{n - 1} $$
which is used to select the best model and as a meta-features on the subset of the task of size 500(chosen arbitrarily).

%%%%%%%%%%%%%%%%%%%%%%%%%%%%%%%%%%%%%%%%%%%%%%%%%%%%%%%%%%%%%%%%%%%%%%%%%%%%%%%%
\section{Possible improvements}

As preprocessing is the most important part in any data analysis task\cite{ten-quick-tips}, it is the first part that
should be improved in automated data analysis. In framework of this project this part did not get deserving attention
because of the lack of authors time, but should be considered as the primary optimisation point of the implemented
algorithm.


%%%%%%%%%%%%%%%%%%%%%%%%%%%%%%%%%%%%%%%%%%%%%%%%%%%%%%%%%%%%%%%%%%%%%%%%%%%%%%%%
\section{Outputs}



%%%%%%%%%%%%%%%%%%%%%%%%%%%%%%%%%%%%%%%%%%%%%%%%%%%%%%%%%%%%%%%%%%%%%%%%%%%%%%%%
\section{Conclusion}

%%%%%%%%%%%%%%%%%%%%%%%%%%%%%%%%%%%%%%%%%%%%%%%%%%%%%%%%%%%%%%%%%%%%%%%%%%%%%%%%
\bibliography{reference}


\end{document}
